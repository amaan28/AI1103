\documentclass[journal,12pt,twocolumn]{IEEEtran}

\usepackage{setspace}
\usepackage{gensymb}
\singlespacing
\usepackage[cmex10]{amsmath}

\usepackage{amsthm}

\usepackage{mathrsfs}
\usepackage{txfonts}
\usepackage{stfloats}
\usepackage{bm}
\usepackage{cite}
\usepackage{cases}
\usepackage{subfig}

\usepackage{longtable}
\usepackage{multirow}

\usepackage{enumitem}
\usepackage{mathtools}
\usepackage{steinmetz}
\usepackage{tikz}
\usepackage{circuitikz}
\usepackage{verbatim}
\usepackage{tfrupee}
\usepackage[breaklinks=true]{hyperref}
\usepackage{graphicx}
\usepackage{tkz-euclide}

\usetikzlibrary{calc,math}
\usepackage{listings}
    \usepackage{color}                                            %%
    \usepackage{array}                                            %%
    \usepackage{longtable}                                        %%
    \usepackage{calc}                                             %%
    \usepackage{multirow}                                         %%
    \usepackage{hhline}                                           %%
    \usepackage{ifthen}                                           %%
    \usepackage{lscape}     
\usepackage{multicol}
\usepackage{chngcntr}

\DeclareMathOperator*{\Res}{Res}

\renewcommand\thesection{\arabic{section}}
\renewcommand\thesubsection{\thesection.\arabic{subsection}}
\renewcommand\thesubsubsection{\thesubsection.\arabic{subsubsection}}

\renewcommand\thesectiondis{\arabic{section}}
\renewcommand\thesubsectiondis{\thesectiondis.\arabic{subsection}}
\renewcommand\thesubsubsectiondis{\thesubsectiondis.\arabic{subsubsection}}

\newcommand{\BEQA}{\begin{eqnarray}}
\newcommand{\EEQA}{\end{eqnarray}}
\newcommand{\define}{\stackrel{\triangle}{=}}
\bibliographystyle{IEEEtran}
\raggedbottom
\setlength{\parindent}{0pt}
\providecommand{\mbf}{\mathbf}
\providecommand{\pr}[1]{\ensuremath{\Pr\left(#1\right)}}
\providecommand{\qfunc}[1]{\ensuremath{Q\left(#1\right)}}
\providecommand{\sbrak}[1]{\ensuremath{{}\left[#1\right]}}
\providecommand{\lsbrak}[1]{\ensuremath{{}\left[#1\right.}}
\providecommand{\rsbrak}[1]{\ensuremath{{}\left.#1\right]}}
\providecommand{\brak}[1]{\ensuremath{\left(#1\right)}}
\providecommand{\lbrak}[1]{\ensuremath{\left(#1\right.}}
\providecommand{\rbrak}[1]{\ensuremath{\left.#1\right)}}
\providecommand{\cbrak}[1]{\ensuremath{\left\{#1\right\}}}
\providecommand{\lcbrak}[1]{\ensuremath{\left\{#1\right.}}
\providecommand{\rcbrak}[1]{\ensuremath{\left.#1\right\}}}
\theoremstyle{remark}
\newtheorem{rem}{Remark}
\newcommand{\sgn}{\mathop{\mathrm{sgn}}}
\providecommand{\abs}[1]{\vert#1\vert}
\providecommand{\res}[1]{\Res\displaylimits_{#1}} 
\providecommand{\norm}[1]{\lVert#1\rVert}
%\providecommand{\norm}[1]{\lVert#1\rVert}
\providecommand{\mtx}[1]{\mathbf{#1}}
\providecommand{\mean}[1]{E[ #1 ]}
\providecommand{\fourier}{\overset{\mathcal{F}}{ \rightleftharpoons}}
%\providecommand{\hilbert}{\overset{\mathcal{H}}{ \rightleftharpoons}}
\providecommand{\system}{\overset{\mathcal{H}}{ \longleftrightarrow}}
	%\newcommand{\solution}[2]{\textbf{Solution:}{#1}}
\newcommand{\solution}{\noindent \textbf{Solution: }}
\newcommand{\cosec}{\,\text{cosec}\,}
\providecommand{\dec}[2]{\ensuremath{\overset{#1}{\underset{#2}{\gtrless}}}}
\newcommand{\myvec}[1]{\ensuremath{\begin{pmatrix}#1\end{pmatrix}}}
\newcommand{\mydet}[1]{\ensuremath{\begin{vmatrix}#1\end{vmatrix}}}
\numberwithin{equation}{subsection}
\makeatletter
\@addtoreset{figure}{problem}
\makeatother
\let\StandardTheFigure\thefigure
\let\vec\mathbf
\renewcommand{\thefigure}{\theproblem}
\def\putbox#1#2#3{\makebox[0in][l]{\makebox[#1][l]{}\raisebox{\baselineskip}[0in][0in]{\raisebox{#2}[0in][0in]{#3}}}}
     \def\rightbox#1{\makebox[0in][r]{#1}}
     \def\centbox#1{\makebox[0in]{#1}}
     \def\topbox#1{\raisebox{-\baselineskip}[0in][0in]{#1}}
     \def\midbox#1{\raisebox{-0.5\baselineskip}[0in][0in]{#1}}

\lstset{
%language=C,
frame=single, 
breaklines=true,
columns=fullflexible
}
\begin{document}
\title{Assignment 3}
\author{Vibhavasu Pasumarti - EP20BTECH11015}
\maketitle
\newpage
\bigskip
\renewcommand{\thefigure}{\theenumi}
\renewcommand{\thetable}{\theenumi}
Download all python codes from 
\begin{lstlisting}
https://github.com/VIB2020/AI1103/blob/main/Assignment%203/code/Assignment%203.py
\end{lstlisting}
%
and latex-tikz codes from 
%
\begin{lstlisting}
https://github.com/VIB2020/AI1103/blob/main/Assignment%203/Assignment%203.pdf
\end{lstlisting}
\section{\Large Problem\\ \large GATE 2015 (EE paper 01 new 2), Q. 27 (Electrical Engg. section)}

Two players A, and B alternately keep rolling a fair dice. The person to get a six first wins the game. Given that player A starts the game, the probability that A wins the game is:\\[5pt]
\begin{enumerate}[label=(\Alph*)]
\onehalfspacing
\Large
    \item  $\frac{5}{11}$
    \item  $\frac{1}{2}$
    \item  $\frac{7}{13}$
    \item  $\frac{6}{11}$
\end{enumerate}

\section{\Large Solution}
\singlespacing
\begin{description}
    \item Let the random variable X denote the win of A.
    \item Given the die is fair.
    \item The probability of getting 6 = $\dfrac{1}{6}$ = p (say)
    \item The probability of NOT getting 6 = $\dfrac{5}{6}$ = q (say)
\end{description}
\begin{center}
Markov chain for the given problem:
\begin{tikzpicture}[font=\sffamily]
    \tikzset{node style/.style={state, 
                                    minimum width=2cm,
                                    line width=1mm,
                                    fill=gray!20!white}}

    \node[node style] at (0, 0)     (a)     {A};
    \node[node style] at (6, 0)     (b)     {B};
    \node[node style] at (3, -5.196) (w) {Winner};
        % Connect the states with arrows
    \draw[every loop,
          auto=right,
          line width=1mm,
          >=latex,
          draw=orange,
          fill=orange]
        
        (a) edge[bend right=20] node {q = $\dfrac{5}{6}$} (b)
        (b) edge[bend right=20] node {q = $\dfrac{5}{6}$} (a)
        (a) edge[bend right=20] node {p = $\dfrac{1}{6}$} (w)
        (b) edge[bend left=20, auto=left] node {p = $\dfrac{1}{6}$} (w)
\end{tikzpicture}
\end{center}
\\[5pt]
\begin{description}
    \item \large  Constraint: A wins the game.
    \normalsize
    \item \pr{\text{A wins on the first throw}} = \pr{$X_1$} = $p$
    \item \pr{\text{A wins on the third throw}} =\pr{$X_3$}= $q^2p$
    \item \pr{\text{A wins on the $(2n + 1)^t^h $ throw}}= \pr{$X_{2n+1}$} =  $q^2^n p$
\end{description}
\begin{align}
    \pr{X} = \sum_{n=1}^{+\infty} \pr{X_i} \\
    \pr{X} = \sum_{n=1}^{+\infty} q^2^n p \\
    = p \sum_{n=1}^{+\infty} (q^2)^n    \\
    = p \left(\dfrac{1}{1 - q^2}\right) = \dfrac{p}{1 - q^2}
    =\dfrac{\dfrac{1}{6}}{1 - \dfrac{25}{36}} = \dfrac{6}{11}
\end{align}
\centering
\Large \pr{\text{A wins the game}} = \pr{X} = \dfrac{6}{11}\\[6pt]
Option D
\end{document}
