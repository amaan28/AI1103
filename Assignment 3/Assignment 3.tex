\documentclass[journal,12pt,twocolumn]{IEEEtran}

\usepackage{setspace}
\usepackage{gensymb}
\onehalfspacing
\usepackage[cmex10]{amsmath}

\usepackage{amsthm}

\usepackage{mathrsfs}
\usepackage{txfonts}
\usepackage{stfloats}
\usepackage{bm}
\usepackage{cite}
\usepackage{cases}
\usepackage{subfig}

\usepackage{longtable}
\usepackage{multirow}

\usepackage{enumitem}
\usepackage{mathtools}
\usepackage{steinmetz}
\usepackage{tikz}
\usetikzlibrary{automata, positioning}
\usepackage{circuitikz}
\usepackage{verbatim}
\usepackage{tfrupee}
\usepackage[breaklinks=true]{hyperref}
\usepackage{graphicx}
\usepackage{tkz-euclide}

\usetikzlibrary{calc,math}
\usepackage{listings}
    \usepackage{color}                                            %%
    \usepackage{array}                                            %%
    \usepackage{longtable}                                        %%
    \usepackage{calc}                                             %%
    \usepackage{multirow}                                         %%
    \usepackage{hhline}                                           %%
    \usepackage{ifthen}                                           %%
    \usepackage{lscape}     
\usepackage{multicol}
\usepackage{chngcntr}

\DeclareMathOperator*{\Res}{Res}

\renewcommand\thesection{\arabic{section}}
\renewcommand\thesubsection{\thesection.\arabic{subsection}}
\renewcommand\thesubsubsection{\thesubsection.\arabic{subsubsection}}

\renewcommand\thesectiondis{\arabic{section}}
\renewcommand\thesubsectiondis{\thesectiondis.\arabic{subsection}}
\renewcommand\thesubsubsectiondis{\thesubsectiondis.\arabic{subsubsection}}


\hyphenation{op-tical net-works semi-conduc-tor}
\def\inputGnumericTable{}                                 %%

\lstset{
%language=C,
frame=single, 
breaklines=true,
columns=fullflexible
}
\begin{document}
\title{Assignment 3}
\author{Vibhavasu Pasumarti - EP20BTECH11015}
\maketitle
\newpage
\bigskip
\renewcommand{\thefigure}{\theenumi}
\renewcommand{\thetable}{\theenumi}
Download all python codes from 
\begin{lstlisting}
https://github.com/VIB2020/AI1103/blob/main/Assignment%203/code/Assignment%203.py
\end{lstlisting}
%
and latex-tikz codes from 
%
\begin{lstlisting}
https://github.com/VIB2020/AI1103/blob/main/Assignment%203/Assignment%203.pdf
\end{lstlisting}
\section{\Large Problem\\ \large GATE 2015 (EE paper 01 new 2), Q. 27 (Electrical Engg. section)}

Two players A, and B alternately keep rolling a fair dice. The person to get a six first wins the game. Given that player A starts the game, the probability that A wins the game is:\\[5pt]
A: \dfrac{5}{11}\hspace{1cm}
B: \dfrac{1}{2}\hspace{1cm}
C: \dfrac{7}{13}\hspace{1cm}
D: \dfrac{6}{11}
\section{\Large Solution}
\begin{description}
    \item Let the random variable X denote the win of A.
    \item Given the die is fair.
    \item The probability of getting 6 = $\dfrac{1}{6}$ = p (say)
    \item The probability of NOT getting 6 = $\dfrac{5}{6}$ = q (say)
    \item \Large
\end{description}

\begin{center}
Markov chain for the given problem:
\newline
\begin{tikzpicture}[font=\sffamily]
    \tikzset{node style/.style={state, 
                                    minimum width=2cm,
                                    line width=1mm,
                                    fill=gray!20!white}}

    \node[node style] at (0, 0)     (a)     {A};
    \node[node style] at (6, 0)     (b)     {B};
    \node[node style] at (3, -5.196) (w) {Winner};
        % Connect the states with arrows
    \draw[every loop,
          auto=right,
          line width=1mm,
          >=latex,
          draw=orange,
          fill=orange]
        
        (a) edge[bend right=20] node {q = $\dfrac{5}{6}$} (b)
        (b) edge[bend right=20] node {q = $\dfrac{5}{6}$} (a)
        (a) edge[bend right=20] node {p = $\dfrac{1}{6}$} (w)
        (b) edge[bend left=20, auto=left] node {p = $\dfrac{1}{6}$} (w)
\end{tikzpicture}
\end{center}
\\[5pt]
\begin{description}
    \item \large  Constraint: A wins the game.
    \normalsize
    \item P(A wins on the first throw) = P($X_1$) = $p$
    \item P(A wins on the third throw) = P($X_3$) = $q^2p$
    \item P(A wins on the $(2n + 1)^t^h $ throw)= P($X_{2n+1}$) = \large $q^2^n p$
\end{description}
\begin{align}
\doublespacing
    P(\text{A wins the game}) = P(X) = \sum_{n=1}^{+\infty} P(X_i) \\
    P(X) = \sum_{n=1}^{+\infty} q^2^n p \\
    = p \sum_{n=1}^{+\infty} (q^2)^n    \\
    = p \left(\dfrac{1}{1 - q^2}\right) = \dfrac{p}{1 - q^2}
    =\dfrac{\dfrac{1}{6}}{1 - \dfrac{25}{36}} = \dfrac{6}{11}
\end{align}
\centering
\Large P(A wins the game) = P(X) = \dfrac{6}{11}\\[6pt]
Option D
\end{document}
